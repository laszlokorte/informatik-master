\documentclass[parskip=half,a4paper]{scrartcl}

\input{../../../structure.tex}

\usepackage{pgfplots}
\usepackage[thinlines]{easytable}

\usetikzlibrary{calc}
\pgfplotsset{compat = newest}
\usetikzlibrary{plotmarks}

\title{Machine Learning Exercise 8}
\author{Laszlo Korte, MtrNr. 6329857}
\date{Universität Hamburg --- \today}

\begin{document}

\maketitle

\section{Plots}

\begin{table}[H]
    \center

    \begin{TAB}(e,1.5cm,1.5cm){|c:c:c:c|}{|c:c:c:c|}
0 &  -13.93 &  -19.91 &  -21.9 \\
-13.93 &  -17.92 &  -19.91 &  -19.91 \\
-19.91 &  -19.91 &  -17.93 &  -13.95 \\
-21.9 &  -19.91 &  -13.95 &  0
\end{TAB}

\caption{Value function after 62 iterations}
\end{table}


\begin{table}[H]
    \center

    \begin{TAB}(e,1.5cm,1.5cm){|c:c:c:c|}{|c:c:c:c|}
- & $\leftarrow$ & $\leftarrow$ & $\leftarrow,\downarrow$ \\
$\uparrow$ & $\uparrow,\leftarrow$ & $\leftarrow,\downarrow$ & $\downarrow$\\
$\uparrow$ & $\uparrow,\rightarrow$ & $\downarrow,\rightarrow$ & $\downarrow$\\
$\uparrow,\rightarrow$ & $\rightarrow$ & $\rightarrow$ & -\\
\end{TAB}

\caption{Policy after 62 iterations}
\end{table}



\end{document}
