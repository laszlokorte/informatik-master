\documentclass[parskip=half,a4paper]{scrartcl}
\usepackage[ngerman]{babel}

%%%%%%%%%%%%%%%%%%%%%%%%%%%%%%%%%%%%%%%%%
% Lachaise Assignment
% Structure Specification File
% Version 1.0 (26/6/2018)
%
% This template originates from:
% http://www.LaTeXTemplates.com
%
% Authors:
% Marion Lachaise & François Févotte
% Vel (vel@LaTeXTemplates.com)
%
% License:
% CC BY-NC-SA 3.0 (http://creativecommons.org/licenses/by-nc-sa/3.0/)
%
%%%%%%%%%%%%%%%%%%%%%%%%%%%%%%%%%%%%%%%%%

%----------------------------------------------------------------------------------------
%   PACKAGES AND OTHER DOCUMENT CONFIGURATIONS
%----------------------------------------------------------------------------------------

\usepackage[utf8]{inputenc} % Required for inputting international characters
\usepackage[T1]{fontenc} % Output font encoding for international characters

\usepackage{amsmath,stmaryrd} % Math packages

\usepackage{enumerate} % Custom item numbers for enumerations
\usepackage[shortlabels]{enumitem}
\usepackage{array}
\usepackage{lineno}
\usepackage{epstopdf}
\usepackage{graphics}

\usepackage[ruled]{algorithm2e} % Algorithms

\usepackage[framemethod=tikz]{mdframed} % Allows defining custom boxed/framed environments

\usepackage{minted}

\usepackage[utopia]{mathdesign}
\usepackage{csquotes}

\usepackage{listings} % File listings, with syntax highlighting
\lstset{
    basicstyle=\ttfamily, % Typeset listings in monospace font
    language=Java,
    numbers=left,
    stepnumber=1,
    showstringspaces=false,
    tabsize=3,
    breaklines=true,
    breakatwhitespace=false,
    frame=single,
    linewidth=\columnwidth
}

%----------------------------------------------------------------------------------------
%   DOCUMENT MARGINS
%----------------------------------------------------------------------------------------

\usepackage{geometry} % Required for adjusting page dimensions and margins

\geometry{
    paper=a4paper, % Paper size, change to letterpaper for US letter size
    top=2.5cm, % Top margin
    bottom=3cm, % Bottom margin
    left=2.5cm, % Left margin
    right=2.5cm, % Right margin
    headheight=14pt, % Header height
    footskip=1.5cm, % Space from the bottom margin to the baseline of the footer
    headsep=1.2cm, % Space from the top margin to the baseline of the header
    %showframe, % Uncomment to show how the type block is set on the page
}

%----------------------------------------------------------------------------------------
%   FONTS
%----------------------------------------------------------------------------------------

%----------------------------------------------------------------------------------------
%   COMMAND  ENVIRONMENT
%----------------------------------------------------------------------------------------

% Usage:
% \begin{commandline}
%   \begin{verbatim}
%       $ ls
%
%       Applications    Desktop ...
%   \end{verbatim}
% \end{commandline}

\mdfdefinestyle{commandline}{
    leftmargin=10pt,
    rightmargin=10pt,
    innerleftmargin=15pt,
    middlelinecolor=black!50!white,
    middlelinewidth=2pt,
    frametitlerule=false,
    backgroundcolor=black!5!white,
    frametitle={Command Line},
    frametitlefont={\normalfont\sffamily\color{white}\hspace{-1em}},
    frametitlebackgroundcolor=black!50!white,
    nobreak,
}

% Define a custom environment for command-line snapshots
\newenvironment{commandline}{
    \medskip
    \begin{mdframed}[style=commandline]
}{
    \end{mdframed}
    \medskip
}

%----------------------------------------------------------------------------------------
%   FILE CONTENTS ENVIRONMENT
%----------------------------------------------------------------------------------------

% Usage:
% \begin{file}[optional filename, defaults to "File"]
%   File contents, for example, with a listings environment
% \end{file}

\mdfdefinestyle{file}{
    innertopmargin=1.6\baselineskip,
    innerbottommargin=0.8\baselineskip,
    topline=false, bottomline=false,
    leftline=false, rightline=false,
    leftmargin=2cm,
    rightmargin=2cm,
    singleextra={%
        \draw[fill=black!10!white](P)++(0,-1.2em)rectangle(P-|O);
        \node[anchor=north west]
        at(P-|O){\ttfamily\mdfilename};
        %
        \def\l{3em}
        \draw(O-|P)++(-\l,0)--++(\l,\l)--(P)--(P-|O)--(O)--cycle;
        \draw(O-|P)++(-\l,0)--++(0,\l)--++(\l,0);
    },
    nobreak,
}

% Define a custom environment for file contents
\newenvironment{file}[1][File]{ % Set the default filename to "File"
    \medskip
    \newcommand{\mdfilename}{#1}
    \begin{mdframed}[style=file]
}{
    \end{mdframed}
    \medskip
}

%----------------------------------------------------------------------------------------
%   NUMBERED QUESTIONS ENVIRONMENT
%----------------------------------------------------------------------------------------

% Usage:
% \begin{question}[optional title]
%   Question contents
% \end{question}

\mdfdefinestyle{question}{
    innertopmargin=1.2\baselineskip,
    innerbottommargin=0.8\baselineskip,
    roundcorner=5pt,
    nobreak,
    singleextra={%
        \draw(P-|O)node[xshift=1em,anchor=west,fill=white,draw,rounded corners=5pt]{%
        Question \theQuestion\questionTitle};
    },
}

\newcounter{Question} % Stores the current question number that gets iterated with each new question

% Define a custom environment for numbered questions
\newenvironment{question}[1][\unskip]{
    \bigskip
    \stepcounter{Question}
    \newcommand{\questionTitle}{~#1}
    \begin{mdframed}[style=question]
}{
    \end{mdframed}
    \medskip
}

%----------------------------------------------------------------------------------------
%   WARNING TEXT ENVIRONMENT
%----------------------------------------------------------------------------------------

% Usage:
% \begin{warn}[optional title, defaults to "Warning:"]
%   Contents
% \end{warn}

\mdfdefinestyle{warning}{
    topline=false, bottomline=false,
    leftline=false, rightline=false,
    nobreak,
    singleextra={%
        \draw(P-|O)++(-0.5em,0)node(tmp1){};
        \draw(P-|O)++(0.5em,0)node(tmp2){};
        \fill[black,rotate around={45:(P-|O)}](tmp1)rectangle(tmp2);
        \node at(P-|O){\color{white}\scriptsize\bf !};
        \draw[very thick](P-|O)++(0,-1em)--(O);%--(O-|P);
    }
}

% Define a custom environment for warning text
\newenvironment{warn}[1][Warning:]{ % Set the default warning to "Warning:"
    \medskip
    \begin{mdframed}[style=warning]
        \noindent{\textbf{#1}}
}{
    \end{mdframed}
}

%----------------------------------------------------------------------------------------
%   INFORMATION ENVIRONMENT
%----------------------------------------------------------------------------------------

% Usage:
% \begin{info}[optional title, defaults to "Info:"]
%   contents
%   \end{info}

\mdfdefinestyle{info}{%
    topline=false, bottomline=false,
    leftline=false, rightline=false,
    nobreak,
    singleextra={%
        \fill[black](P-|O)circle[radius=0.4em];
        \node at(P-|O){\color{white}\scriptsize\bf i};
        \draw[very thick](P-|O)++(0,-0.8em)--(O);%--(O-|P);
    }
}

% Define a custom environment for information
\newenvironment{info}[1][Info:]{ % Set the default title to "Info:"
    \medskip
    \begin{mdframed}[style=info]
        \noindent{\textbf{#1}}
}{
    \end{mdframed}
}


\newdimen\longformulasindent
\newenvironment{longformulas}
 {\global\longformulasindent=0pt
  \def\>{\global\advance\longformulasindent2em\relax\hspace{2em}}%
  \def\<{\global\advance\longformulasindent-2em\relax\hspace{-2em}}%
  \renewcommand{\arraystretch}{1.2}% some more room
  \begin{array}{@{}>{\displaystyle\hspace{\longformulasindent}}l@{}}}
 {\end{array}}

\newcommand{\code}{\texttt}


\title{Stochastik Blatt 1}

\author{Mtr.-Nr. 6329857}

\date{Universität Hamburg --- \today}

\begin{document}

\maketitle

\section*{Aufgabe 1}

\begin{enumerate}[(a)]
\item Wenigstens eines der drei Ereignisse tritt ein:\\ $A \cup B \cup C$
\item Höchstens eines der drei Ereignisse tritt ein:\\ $
(A \cap B^C \cap C^C) \cup
(A^C \cap B \cap C^C) \cup
(A^C \cap B^C \cap C) \cup
(A^C \cap B^C \cap C^C)
$
\end{enumerate}

\section*{Aufgabe 2}

\begin{enumerate}[(a)]
\item $\Omega = \setc[\Big]{ (w_1, w_2, w_3)}{w_1,w_2,w_3 \in \left\{K,Z\right\}}$

\item $A = \setc[\Big]{ (w_1, w_2, w_3)}{(w_1,w_2,w_3) \in \Omega \land w_1 = w_2}$ \\ $B = \setc[\Big]{ (w_1, w_2, w_3)}{(w_1,w_2,w_3) \in \Omega \land w_2 = w_3}$

\item $A \cap B = \setc[\Big]{ (w, w, w)}{(w,w,w) \in \Omega}$: Alle drei Münzewürfe landen auf der selben Seite \\
$A \setminus B = \setc[\Big]{ (w, w, w_3)}{(w,w,w_3) \in \Omega \land w \neq w_3}$: Die ersten beiden Münzwürfe landen auf der selben Seite, der dritte aber auf einer anderen.

\item
\begin{align*}
\forall w_1,w_2,w_3 \in \left\{K, Z\right\}&:
P\left(\left\{\left(w_1, w_2, w_3\right)\right\}\right) = \frac{1}{8}\\
P\left(A\right) &= P\left(\left\{\left(K,K,K\right)\right\}\right) + P\left(\left\{\left(K,K,Z\right)\right\}\right) + P\left(\left\{\left(Z,Z,K\right)\right\}\right) + P\left(\left\{\left(Z,Z,Z\right)\right\}\right) \\&= \frac{4}{8}\\
P\left(A \setminus B\right) &= P\left(\left\{\left(K,K,Z\right)\right\}\right) + P\left(\left\{\left(Z,Z,K\right)\right\}\right) \\&= \frac{2}{8}
\end{align*}


\end{enumerate}

\section*{Aufgabe 3}

\begin{enumerate}[(a)]
\item
\begin{align}
    P(\Omega) &= 1\\
\Omega \cap \emptyset &= \emptyset\\
\Omega \cup \emptyset &= \Omega\\
P(\Omega \cup \emptyset) &= P(\Omega) + P(\emptyset)\\
1 &= 1 + P(\emptyset)\\
P(\emptyset) &= 0
\end{align}

\item

Zu zeigen für alle $n$:
\begin{align}
\forall A_1,\dots,A_n \subset \Omega \text{ sind disjunkt }:
P(\bigcup\limits_{i=1}^{n} A_i) = \sum\limits_{i=1}^{n} P(A_i)
\end{align}

Induktionsanfang: Für $n = 1$ gilt:
\begin{align}
A_1 \subset \Omega \text{ ist disjunkt }
:
P(\bigcup\limits_{i=1}^{n=1} A_i) &= \sum\limits_{i=1}^{n=1} P(A_i)\\
\text{ denn } P(A_1) &= P(A_1)
\end{align}


Indunktionsschritt: Wenn für ein beliebig festes $n$ gilt:
\begin{align}
\forall A_1,\dots,A_n \subset \Omega \text{ sind disjunkt }:
P(\bigcup\limits_{i=1}^{n} A_i) = \sum\limits_{i=1}^{n} P(A_i)
\end{align}

Und ein weiteres Ereignis $A_{n+1}$ sei disjunkt zu allen Ereignissen $A_1,\dots,A_n$, dann gilt auch:

\begin{align}
P((\bigcup\limits_{i=1}^{n} A_i) \cup A_{n+1}) = (\sum\limits_{i=1}^{n} P(A_i)) + P(A_{n+1})
\end{align}

Durch jeweilig zusammenführung der Summen und Vereinigung gilt entsprend auch:

\begin{align}
\forall A_1,\dots,A_n,A_{n+1} \subset \Omega \text{ sind disjunkt }:
P(\bigcup\limits_{i=1}^{n+1} A_i) = \sum\limits_{i=1}^{n+1} P(A_i)
\end{align}

Indunktionsabschluss: Da wir gezeigt haben, dass die Aussage falls sie für ein beliebiges $n$ gilt auch für $n+1$ gilt, und dass sie für $n=1$ gilt, gilt sie per Indunktion für alle $n \in \mathbb{N}$


\item
\begin{align}
    A &\subset \Omega\\
    P(\Omega) &= 1\\
    A \cap A^C &= \emptyset\\
    A \cup A^C & = \Omega\\
    P(A) + P(A^C) &= P(\Omega) = 1\\
    P(A^C) &= 1 - P(A)
\end{align}




\item

\begin{align}
    A &\subset B \subset \Omega:\\
    (B \setminus A) \cap A &= \emptyset\\
    P(B\setminus A) + P(A) &= P((B\setminus A) \cup A)\\
    P(B\setminus A) + P(A) &= P(B)\\
    P(B\setminus A) &= P(B) - P(A)
\end{align}
\end{enumerate}


\end{document}
