\documentclass[parskip=half,a4paper]{scrartcl}
\usepackage[ngerman]{babel}

\input{../../../structure.tex}

\title{Korrektur Blatt 3 Teil 1 für 6435479}

\author{Korrigiert von Mtr.-Nr. 6329857}

\date{Universität Hamburg --- \today}

\begin{document}

\maketitle % Print the title

\section*{Aufgabe 6}

\begin{enumerate}
    \item Die Funktionsweise des Algorithmus ist nicht korrekt beschrieben: In der Beschreibung von Increment wird zwischen 4 fällen von graden und ungrade Zahlen unterschieden. Da der Algorithmus nie explizit auf das niedrigste Bit zugreift, ist der Zusammenhang zwischen Algorithmus und Beschreibung nicht klar. Außerdem wurde die Symmetrie zwischen der Increment und der Decrement Operation in der Beschreibung nicht erwähnt.

    \item Es wurden passende Kosten entsprechend der Accounting-Methode gewählt. Allerdings wurde nicht begründet warum für $\text{RESET}_N$ Kosten von $0$ gewählt werden dürfen.

    \item Die Potentialfunktion wurde nicht explizit angebeben. Es wurde nicht verständlich gezeigt, dass es sich um eine erlaubte Potentialfunktion handelt. Das Ergebnis scheint richtig zu sein, aber der Weg nicht nicht genau nochvollziehbar.
\end{enumerate}

\section*{Aufgabe 7}

\begin{enumerate}
    \item Es ist wieder nicht ganz klar wie argumentiert wurde. Aufällig ist, dass der Betragsoperator aus der gegebenen Potentialfunktion weggelassen wurde ohne eine Fallunterscheidung durchzuführen.
\end{enumerate}

\end{document}
