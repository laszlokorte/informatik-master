\documentclass[parskip=half,a4paper]{scrartcl}
\usepackage[ngerman]{babel}

\input{../../../structure.tex}

\title{Korrektur Blatt 5 Teil 2 für 6533594}

\author{Korrigiert von Mtr.-Nr. 6329857}

\date{Universität Hamburg --- \today}

\begin{document}

\maketitle % Print the title

\section{Aufgabe 14}

Volle Punktzahl

\begin{itemize}
    \item Die Fehlerhaftigkeit des Verfahrens wurde korrekt begründet, inkl. Gegenbeispiel
    \item Ein korrektes Verfahren wurde vorgestellt und nachvollziehbar erläutert
    \item Die Einhaltung der Laufzeitschranke wurde nicht explizit begründet, ergibt sich aber offensichtlich daraus, dass Bellman-Ford auf einfache Weise modifiziert wurde.
    \item Die Korrektheit des Verfahrens wurde nachvollziehbar begründet, da ja sie Ursache für den korrekt analysierten Fehler angegeben wurde.
\end{itemize}

\section{Aufgabe 15}


\begin{itemize}
    \item Ein korrekter Algorithmus wurde nachvollziebar in Worten beschrieben
    \item Laufzeit wurde nicht explizit begründet, allerdings wurde einfach ein Dijkstra benutzt, woraus sich die Laufzeit offensichtlich ergibt.
\end{itemize}



\end{document}
