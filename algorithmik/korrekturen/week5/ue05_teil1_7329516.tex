\documentclass[parskip=half,a4paper]{scrartcl}
\usepackage[ngerman]{babel}

\input{../../../structure.tex}

\title{Korrektur Blatt 5 Teil 1 für 7329516}

\author{Korrigiert von Mtr.-Nr. 6329857}

\date{Universität Hamburg --- \today}

\begin{document}

\maketitle % Print the title

\section{Aufgabe 13}

\subsection{Teil 1}

\enquote{Somit sind $V_B$ und $E_B$ Teilmengen von $V$ und $E$.}(Zeile 3) ist nicht exakt korrekt. $V_B$ und $E_B$ sind werden zwar aus $V$ und $E$ konstruiert, allerdings werden ganze Komponenten in $G$ zu Knoten in $G_B$ zusammengefasst und nicht nur Knoten aus $G$ ausgewählt. Die Kanten zwischen Komponentenknoten und Gelenkknoten sind in $G$ nicht anthalten.

\enquote{[\dots]dass in $E_B$ keine Kanten sein muss, [\dots]} (Zeile 5) --- Sprachlogisch muss anstelle des \enquote{muss} ein \enquote{darf} stehen: Es darf keine solche Kante existieren.

Der Beweis der Zyklenfreiheit scheint etwas knapp zu sein und nur Pfade der Länge 2 zu betrachten. Es wird nicht ganz klar warum es keine Zyklen geben kann.

\subsection{Teil 2}

\begin{itemize}
    \item Brücken werden korrekt gefunden.
    \item Die Laufzeit wurde nicht explzit angeben und entsprechend auch nicht Begründet. Allerdings wurde ein aus der Vorlesung bekannter Algorithmus nur leicht modifiert, woraus sich die Laufzeit offensichtlich ergibt.
\end{itemize}

\end{document}
