\documentclass[parskip=half,a4paper]{scrartcl}
\usepackage[ngerman]{babel}

\input{../../../structure.tex}

\title{Korrektur Blatt 1 Aufgabe 2 für 7027218}

\author{Mtr.-Nr. 6329857}

\date{Universität Hamburg --- \today}

\begin{document}

\maketitle % Print the title


\section*{Schleifeninvariante Modulo}

Volle Punktzahl:

\begin{enumerate}
    \item Die Schleifeninvariant ist korrekt. \checkmark
    \item Der Beweis ist gegliedert in vor der Schleife, während der Schleife und nach der Schleife. \checkmark
    \item Die drei Schritte sind jeweils korrekt gezeigt.  \checkmark
\end{enumerate}


\section*{Primzahl}

Als Invariante wurde angegeben: \enquote{Alle Zahlen, \textbf{die Primfaktoren < i} haben wurden als nicht Primzahl markiert.}

Es müsste aber sein \enquote{\textbf{Alle Zahlen < i}, die Primfaktoren haben wurden als nicht Primzahl markiert}, denn zu Beginn sind ja noch alle Zahlen als Primzahl markiert, wodurch die Invariante so gar nicht stimmen kann.

Außerdem fehlt für einen Korrektheitsbeweis die Rückrichtung der Invariante, nämlich dass alle Primzahlen < i auch als solche markiert wurden.

Durch die fehlerhafte und unvollständige Invariante funktionieren die folgenden Beweisschritte natürlich nicht.

\begin{enumerate}
    \item Die Schleifeninvariant ist korrekt. (nein: $-1$ Punkt)
    \item Der Beweis ist gegliedert in vor der Schleife, während der Schleife und nach der Schleife. \checkmark
    \item Die drei Schritte sind jeweils korrekt gezeigt. Folgefehler?
\end{enumerate}




\end{document}
