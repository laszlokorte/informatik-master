\documentclass[parskip=half,a4paper]{scrartcl}
\usepackage[ngerman]{babel}

\input{../../../structure.tex}

\title{Korrektur Blatt 7 Teil 1 für 7030197}

\author{Korrigiert von Mtr.-Nr. 6329857}

\date{Universität Hamburg --- \today}

\begin{document}

\maketitle % Print the title


\section{Aufgabe 25}

Die Angegebene Gewichtsfunktion scheint unnötig kompliziert, besonders durch den $m+1$ Exponenten. Es scheint, dass wenn $\forall e: c(e) < \frac{1}{m}$, $x_i > 1$ sein kann, z.B. mit $m=2$ und $c(e) = 0.01$ $\frac{0.01}{2 * (0.01) ^ 3} = 5000 > 1$.

Der Beweis für Behauptung 1 ist rückwärts, weil das, was zu beweisen ist, einfach angenommen zu werden scheint.
s
\section{Aufgabe 26}

\begin{itemize}
    \item Das angegebene Verfahren weicht von der Musterlösung ab, wurde aber verständlich erklärt und scheint zu funktionieren, leider nur natürlichsprachlich
    \item Die Laufzeit nicht nicht optimal
    \item Auf Grund technischer Probleme wurde der Pseudocode nicht lesbar angegeben
\end{itemize}

\end{document}
