\documentclass[parskip=half,a4paper]{scrartcl}
\usepackage[ngerman]{babel}

\input{../../../structure.tex}

\title{Korrektur Blatt 7 Teil 2 für 7052484}

\author{Korrigiert von Mtr.-Nr. 6329857}

\date{Universität Hamburg --- \today}

\begin{document}

\maketitle % Print the title

\section{Aufgabe 27}

\subsection{Teil 1}


\begin{itemize}
    \item Die Grundidee die Elemete in die Blätter zu rotieren ist richtig
    \item Allerdings wird überhaupt nicht auf die Zugriffszeiten eingegangen, sondern stattdessen mit MIN, MIN2 und MAX gearbeitet. Es wird nicht erklärt warum. Die größe der Element-Werte sollten für die Anordnung keine Rolle spielen.
\end{itemize}

\subsection{Teil 2}

\begin{itemize}
    \item MOVE-TO-ROOT(5) wurde gefunden
    \item Eine korrekte Sequenz von mehreren MOVE-TO-ROOT Operationen in der alle Knoten vorkommen wurde gefunden.
\end{itemize}

\end{document}
