\documentclass[parskip=half,a4paper]{scrartcl}
\usepackage[ngerman]{babel}

%%%%%%%%%%%%%%%%%%%%%%%%%%%%%%%%%%%%%%%%%
% Lachaise Assignment
% Structure Specification File
% Version 1.0 (26/6/2018)
%
% This template originates from:
% http://www.LaTeXTemplates.com
%
% Authors:
% Marion Lachaise & François Févotte
% Vel (vel@LaTeXTemplates.com)
%
% License:
% CC BY-NC-SA 3.0 (http://creativecommons.org/licenses/by-nc-sa/3.0/)
%
%%%%%%%%%%%%%%%%%%%%%%%%%%%%%%%%%%%%%%%%%

%----------------------------------------------------------------------------------------
%   PACKAGES AND OTHER DOCUMENT CONFIGURATIONS
%----------------------------------------------------------------------------------------

\usepackage[utf8]{inputenc} % Required for inputting international characters
\usepackage[T1]{fontenc} % Output font encoding for international characters

\usepackage{amsmath,stmaryrd} % Math packages

\usepackage{enumerate} % Custom item numbers for enumerations
\usepackage[shortlabels]{enumitem}
\usepackage{array}
\usepackage{lineno}
\usepackage{epstopdf}
\usepackage{graphics}

\usepackage[ruled]{algorithm2e} % Algorithms

\usepackage[framemethod=tikz]{mdframed} % Allows defining custom boxed/framed environments

\usepackage{minted}

\usepackage[utopia]{mathdesign}
\usepackage{csquotes}

\usepackage{listings} % File listings, with syntax highlighting
\lstset{
    basicstyle=\ttfamily, % Typeset listings in monospace font
    language=Java,
    numbers=left,
    stepnumber=1,
    showstringspaces=false,
    tabsize=3,
    breaklines=true,
    breakatwhitespace=false,
    frame=single,
    linewidth=\columnwidth
}

%----------------------------------------------------------------------------------------
%   DOCUMENT MARGINS
%----------------------------------------------------------------------------------------

\usepackage{geometry} % Required for adjusting page dimensions and margins

\geometry{
    paper=a4paper, % Paper size, change to letterpaper for US letter size
    top=2.5cm, % Top margin
    bottom=3cm, % Bottom margin
    left=2.5cm, % Left margin
    right=2.5cm, % Right margin
    headheight=14pt, % Header height
    footskip=1.5cm, % Space from the bottom margin to the baseline of the footer
    headsep=1.2cm, % Space from the top margin to the baseline of the header
    %showframe, % Uncomment to show how the type block is set on the page
}

%----------------------------------------------------------------------------------------
%   FONTS
%----------------------------------------------------------------------------------------

%----------------------------------------------------------------------------------------
%   COMMAND  ENVIRONMENT
%----------------------------------------------------------------------------------------

% Usage:
% \begin{commandline}
%   \begin{verbatim}
%       $ ls
%
%       Applications    Desktop ...
%   \end{verbatim}
% \end{commandline}

\mdfdefinestyle{commandline}{
    leftmargin=10pt,
    rightmargin=10pt,
    innerleftmargin=15pt,
    middlelinecolor=black!50!white,
    middlelinewidth=2pt,
    frametitlerule=false,
    backgroundcolor=black!5!white,
    frametitle={Command Line},
    frametitlefont={\normalfont\sffamily\color{white}\hspace{-1em}},
    frametitlebackgroundcolor=black!50!white,
    nobreak,
}

% Define a custom environment for command-line snapshots
\newenvironment{commandline}{
    \medskip
    \begin{mdframed}[style=commandline]
}{
    \end{mdframed}
    \medskip
}

%----------------------------------------------------------------------------------------
%   FILE CONTENTS ENVIRONMENT
%----------------------------------------------------------------------------------------

% Usage:
% \begin{file}[optional filename, defaults to "File"]
%   File contents, for example, with a listings environment
% \end{file}

\mdfdefinestyle{file}{
    innertopmargin=1.6\baselineskip,
    innerbottommargin=0.8\baselineskip,
    topline=false, bottomline=false,
    leftline=false, rightline=false,
    leftmargin=2cm,
    rightmargin=2cm,
    singleextra={%
        \draw[fill=black!10!white](P)++(0,-1.2em)rectangle(P-|O);
        \node[anchor=north west]
        at(P-|O){\ttfamily\mdfilename};
        %
        \def\l{3em}
        \draw(O-|P)++(-\l,0)--++(\l,\l)--(P)--(P-|O)--(O)--cycle;
        \draw(O-|P)++(-\l,0)--++(0,\l)--++(\l,0);
    },
    nobreak,
}

% Define a custom environment for file contents
\newenvironment{file}[1][File]{ % Set the default filename to "File"
    \medskip
    \newcommand{\mdfilename}{#1}
    \begin{mdframed}[style=file]
}{
    \end{mdframed}
    \medskip
}

%----------------------------------------------------------------------------------------
%   NUMBERED QUESTIONS ENVIRONMENT
%----------------------------------------------------------------------------------------

% Usage:
% \begin{question}[optional title]
%   Question contents
% \end{question}

\mdfdefinestyle{question}{
    innertopmargin=1.2\baselineskip,
    innerbottommargin=0.8\baselineskip,
    roundcorner=5pt,
    nobreak,
    singleextra={%
        \draw(P-|O)node[xshift=1em,anchor=west,fill=white,draw,rounded corners=5pt]{%
        Question \theQuestion\questionTitle};
    },
}

\newcounter{Question} % Stores the current question number that gets iterated with each new question

% Define a custom environment for numbered questions
\newenvironment{question}[1][\unskip]{
    \bigskip
    \stepcounter{Question}
    \newcommand{\questionTitle}{~#1}
    \begin{mdframed}[style=question]
}{
    \end{mdframed}
    \medskip
}

%----------------------------------------------------------------------------------------
%   WARNING TEXT ENVIRONMENT
%----------------------------------------------------------------------------------------

% Usage:
% \begin{warn}[optional title, defaults to "Warning:"]
%   Contents
% \end{warn}

\mdfdefinestyle{warning}{
    topline=false, bottomline=false,
    leftline=false, rightline=false,
    nobreak,
    singleextra={%
        \draw(P-|O)++(-0.5em,0)node(tmp1){};
        \draw(P-|O)++(0.5em,0)node(tmp2){};
        \fill[black,rotate around={45:(P-|O)}](tmp1)rectangle(tmp2);
        \node at(P-|O){\color{white}\scriptsize\bf !};
        \draw[very thick](P-|O)++(0,-1em)--(O);%--(O-|P);
    }
}

% Define a custom environment for warning text
\newenvironment{warn}[1][Warning:]{ % Set the default warning to "Warning:"
    \medskip
    \begin{mdframed}[style=warning]
        \noindent{\textbf{#1}}
}{
    \end{mdframed}
}

%----------------------------------------------------------------------------------------
%   INFORMATION ENVIRONMENT
%----------------------------------------------------------------------------------------

% Usage:
% \begin{info}[optional title, defaults to "Info:"]
%   contents
%   \end{info}

\mdfdefinestyle{info}{%
    topline=false, bottomline=false,
    leftline=false, rightline=false,
    nobreak,
    singleextra={%
        \fill[black](P-|O)circle[radius=0.4em];
        \node at(P-|O){\color{white}\scriptsize\bf i};
        \draw[very thick](P-|O)++(0,-0.8em)--(O);%--(O-|P);
    }
}

% Define a custom environment for information
\newenvironment{info}[1][Info:]{ % Set the default title to "Info:"
    \medskip
    \begin{mdframed}[style=info]
        \noindent{\textbf{#1}}
}{
    \end{mdframed}
}


\newdimen\longformulasindent
\newenvironment{longformulas}
 {\global\longformulasindent=0pt
  \def\>{\global\advance\longformulasindent2em\relax\hspace{2em}}%
  \def\<{\global\advance\longformulasindent-2em\relax\hspace{-2em}}%
  \renewcommand{\arraystretch}{1.2}% some more room
  \begin{array}{@{}>{\displaystyle\hspace{\longformulasindent}}l@{}}}
 {\end{array}}

\newcommand{\code}{\texttt}


\title{Algorithmik Blatt 3 Teil 1}

\author{Mtr.-Nr. 6329857}

\date{Universität Hamburg --- \today}

\begin{document}

\maketitle % Print the title

\linenumbers


\section*{Aufgabe 6}

\subsection*{Was macht der Algorithmus?}

Der Wert des Zählers wird durch die Differenz zweier Teilzähler $P$ und $N$ gebildet. Wir haben bereits gesehen, dass ein einzelner Zähler in armotisierter worst-case Zeit $\mathcal{O}(1)$ inkrementiert werden kann. Das Hinzunehmen der \textsc{Decrement} Operation verschlechtert die armotisierte Laufzeit jedoch auf $\mathcal{O}(n)$, weil im worst-case eine Sequenz aus abwechselnen \textsc{Increment} und \textsc{Decrement} dazu führen, dass in jeder Operation alle $n$ Bits gewechselt werden müssen.

Der neue Algorithmus umgeht das Problem, indem die \textsc{Decrement} Operation einfach als \textsc{Increment} auf einem zweiten Zähler agiert. Nun sind abwechselnde \textsc{Increment} und \textsc{Decrement} einfach nur \textsc{Increments} auf unterschiedlichen Zählern und die Argumentation für die armortisierte $\mathcal{O}(1)$ Laufzeit von \textsc{Increment} lässt sich auf $\left\{\textsc{Increment}, \textsc{Decrement}\right\}$ übertragen.

Allerdings muss der Algorithmus einen Sonderheit beachten: Jedes Bit darf nur in einem der beiden Zähler auf 1 stehen. Das ist wichtig, um ungewollte mehrfache Repräsentation von Werten zu verhindern. Ansonsten stellten z.B. $(P=00, N=00)$, $(P=01, N=01)$, $(P=10, N=10)$ und $(P=11, N=11)$ alle den Gesamtwert 0 dar. Dadurch wäre bei gleicher Anzahl Bits der Wertebereich eingeschränkt. Immer wenn der Algorithmus ein Bit auf $1$ setzen will, muss er zunächst prüfen, ob das Partnerbit im anderen Zähler schon auf $1$ steht und falls ja, dieses dort stattdessen auf $0$ setzen.

\subsection*{Accounting-Methode}

Die relevanten Operationen der \textsc{Increment} und \textsc{Decrement} Funktion sind die Schreibzugriffe auf die jeweils $log(n)$ breiten Zähler $P$ und $N$. Die Operation die ein Bit $i$ auf 0 setzt nennen wir $\textsc{RESET}_P(i)$ bzw. $\textsc{RESET}_N(i)$. Die Operation die ein Bit $i$ auf 1 setzt nennen wir $\textsc{SET}_P(i)$ bzw. $\textsc{SET}_N(i)$.

Die Laufzeit von \textsc{Increment} ist proportional zur Anzahl der ausgeführten $\textsc{RESET}_P(i)$ Operationen, denn diese kommt unbedingt genau einmal in der einzigen Schleife vor. Entsprechend ist die Laufzeit von \textsc{Decrement} ist proportional zur Anzahl der ausgeführten $\textsc{RESET}_N(i)$ Operationen.

Zu Beginn seien alle Bits auf 0 gesetzt. $\textsc{RESET}_P(i)$ wird nur durchgeführt, wenn Bit $i$ auf 1 gesetzt ist. Dies kann nur der Fall sein, wenn vorher ein $\textsc{SET}_P(i)$ durchgeführt wurde. Analog gilt dies für $\textsc{RESET}_N(i)$.

Also gilt (wobei $\#(\textit{Op})$ die Anzahl der Ausführungen von $\textit{Op}$ ist):

\begin{equation}
\#(\textsc{RESET}_N) \le \#(\textsc{SET}_N)
\end{equation}

Wir können im Accounting die \textsc{RESET} Operationen also die Kosten für die \textsc{SET} Operationen übernehmen lassen. Somit veranschlagen wir:

\begin{equation}
\begin{aligned}
T_{SET_P} &= 2 \\
T_{SET_N} &= 2 \\
T_{RESET_P} &= 0 \\
T_{RESET_N} &= 0 \\
\end{aligned}
\end{equation}

Für $n$ ausführungen von \textsc{Increment} ergibt sich eine gesamte Laufzeit von:

\begin{equation}
\begin{aligned}
T_{INC}(p) &= p \cdot (T_{SET_P} + log(p) \cdot T_{RESET_P} + T_{RESET_N}) \\
 &= p \cdot T_{SET_P}
\end{aligned}
\end{equation}

Für \textsc{Decrement} entsprechend

\begin{equation}
\begin{aligned}
T_{DEC}(q) &= q \cdot (T_{SET_N} + log(q) \cdot T_{RESET_N} + T_{RESET_P}) \\
& \text{weil  $T_{RESET_P} = T_{RESET_N} = 0$} \\
T_{DEC}(q)  &= q \cdot T_{SET_N}
\end{aligned}
\end{equation}

Für $n$ ausgeführte \textsc{Increment} und $m$ ausgeführte \textsc{Increment} ergibt sich:

\begin{equation}
\begin{aligned}
T_{INC}(p) + T_{DEC}(q) &= p \cdot T_{SET_P} + q \cdot T_{SET_N}\\
T_{SET_P} &= T_{SET_N} = 2 \\
T_{INC}(p) + T_{DEC}(q) &= (p+q) \cdot 2
\end{aligned}
\end{equation}

Da \textsc{Increment} und \textsc{Decrement} die einzig erlaubten Operationen sind ergibt sich für jede beliebe Sequenz aus $n = (p+q)$ Operationen eine Laufzeit von $\mathcal{O}(n\cdot 2) = \mathcal{O}(n)$ und damit einer armortisierte worst-case Laufzeit von $\mathcal{O}(1)$.

\subsection*{Potential-Funktion}

Die tatsächlichen Kosten $c_i$ der \textsc{Increment} Operation von Schritt $i$ einer Sequenz von Operationen ergeben sich aus der Anzahl der Teiloperationen $\textsc{RESET}_P$, $\textsc{RESET}_N$ und $\textsc{SET}_P$ in Schritt $i$:

$$
c_i = \#_i(\textsc{RESET}_P) + \#_i(\textsc{RESET}_N) + \#_i(\textsc{SET}_P)
$$

Wobei $\#_i(\textit{Op})$ die Anzahl der ausgeführten Teiloperationen $\textit{Op}$ in Schritt $i$ der Sequenz ist. Die armortisierten Kosten der $i$-ten Operation sind:

\begin{equation}
c_i' = c_i + \phi_i(P,N) - \phi_{i-1}(P,N)
\end{equation}

Wählen wir als Potentialfunktion $\phi$ die Gesamtzahl der in $P$ und $N$ auf 1 gesetzten Bits, ergibt sich:

\begin{equation}
\phi_i(P,N) - \phi_{i-1}(P,N) = \#_i(\textsc{SET}_P) - \#_i(\textsc{RESET}_P) - \#_i(\textsc{RESET}_N)
\end{equation}

Das Potential kann nie negativ sein, da die Anzahl der auf $1$ gesetzten Bits nicht negativ sein kann.

Fügen wir die drei Gleichungne zusammen ergibt sich:

\begin{equation}
\begin{aligned}
c_i' &= \#_i(\textsc{RESET}_P) + \#_i(\textsc{RESET}_N) + \#_i(\textsc{SET}_P) \\ &+ \#_i(\textsc{SET}_P) - \#_i(\textsc{RESET}_P) - \#_i(\textsc{RESET}_N)\\
&= 2 \cdot \#_i(\textsc{SET}_P)\\
\end{aligned}
\end{equation}

Da $\textsc{SET}_P$ nur höchstens einmal pro \textsc{Increment} vorkommt, ergibt sich $c_i' = 2 \le 2$ als armortisierte Kosten der \textsc{Increment} Operation.

Analog funktioniert die Argumentation für Decrement:

\begin{equation}
\begin{aligned}
c_i' &= c_i + \phi_i(P,N) - \phi_{i-1}(P,N)\\
c_i &= \#_i(\textsc{RESET}_N) + \#_i(\textsc{RESET}_P) + \#_i(\textsc{SET}_N)\\
\phi_i(P,N) - \phi_{i-1}(P,N) &= \#_i(\textsc{SET}_N) - \#_i(\textsc{RESET}_N) - \#_i(\textsc{RESET}_P)\\
c_i' &= \#_i(\textsc{RESET}_N) + \#_i(\textsc{RESET}_P) + \#_i(\textsc{SET}_N) \\ &+ \#_i(\textsc{SET}_N) - \#_i(\textsc{RESET}_N) - \#_i(\textsc{RESET}_P)\\
&= 2 \cdot \#_i(\textsc{SET}_N)\\
c_i' &= 2 \le 2
\end{aligned}
\end{equation}

Für eine Sequenz aus $n$ beliebigen Operationen aus der Menge $\{\textsc{Increment}, \textsc{Decrement}\}$ ergibt sich $T(n) = 2 \cdot n$, also eine armortisierte worst-case Laufzeit von $\mathcal{O}(1)$.

\section*{Aufgabe 7}


\end{document}
