\documentclass[parskip=half,a4paper]{scrartcl}

%%%%%%%%%%%%%%%%%%%%%%%%%%%%%%%%%%%%%%%%%
% Lachaise Assignment
% Structure Specification File
% Version 1.0 (26/6/2018)
%
% This template originates from:
% http://www.LaTeXTemplates.com
%
% Authors:
% Marion Lachaise & François Févotte
% Vel (vel@LaTeXTemplates.com)
%
% License:
% CC BY-NC-SA 3.0 (http://creativecommons.org/licenses/by-nc-sa/3.0/)
%
%%%%%%%%%%%%%%%%%%%%%%%%%%%%%%%%%%%%%%%%%

%----------------------------------------------------------------------------------------
%   PACKAGES AND OTHER DOCUMENT CONFIGURATIONS
%----------------------------------------------------------------------------------------

\usepackage[utf8]{inputenc} % Required for inputting international characters
\usepackage[T1]{fontenc} % Output font encoding for international characters

\usepackage{amsmath,stmaryrd} % Math packages

\usepackage{enumerate} % Custom item numbers for enumerations
\usepackage[shortlabels]{enumitem}
\usepackage{array}
\usepackage{lineno}
\usepackage{epstopdf}
\usepackage{graphics}

\usepackage[ruled]{algorithm2e} % Algorithms

\usepackage[framemethod=tikz]{mdframed} % Allows defining custom boxed/framed environments

\usepackage{minted}

\usepackage[utopia]{mathdesign}
\usepackage{csquotes}

\usepackage{listings} % File listings, with syntax highlighting
\lstset{
    basicstyle=\ttfamily, % Typeset listings in monospace font
    language=Java,
    numbers=left,
    stepnumber=1,
    showstringspaces=false,
    tabsize=3,
    breaklines=true,
    breakatwhitespace=false,
    frame=single,
    linewidth=\columnwidth
}

%----------------------------------------------------------------------------------------
%   DOCUMENT MARGINS
%----------------------------------------------------------------------------------------

\usepackage{geometry} % Required for adjusting page dimensions and margins

\geometry{
    paper=a4paper, % Paper size, change to letterpaper for US letter size
    top=2.5cm, % Top margin
    bottom=3cm, % Bottom margin
    left=2.5cm, % Left margin
    right=2.5cm, % Right margin
    headheight=14pt, % Header height
    footskip=1.5cm, % Space from the bottom margin to the baseline of the footer
    headsep=1.2cm, % Space from the top margin to the baseline of the header
    %showframe, % Uncomment to show how the type block is set on the page
}

%----------------------------------------------------------------------------------------
%   FONTS
%----------------------------------------------------------------------------------------

%----------------------------------------------------------------------------------------
%   COMMAND  ENVIRONMENT
%----------------------------------------------------------------------------------------

% Usage:
% \begin{commandline}
%   \begin{verbatim}
%       $ ls
%
%       Applications    Desktop ...
%   \end{verbatim}
% \end{commandline}

\mdfdefinestyle{commandline}{
    leftmargin=10pt,
    rightmargin=10pt,
    innerleftmargin=15pt,
    middlelinecolor=black!50!white,
    middlelinewidth=2pt,
    frametitlerule=false,
    backgroundcolor=black!5!white,
    frametitle={Command Line},
    frametitlefont={\normalfont\sffamily\color{white}\hspace{-1em}},
    frametitlebackgroundcolor=black!50!white,
    nobreak,
}

% Define a custom environment for command-line snapshots
\newenvironment{commandline}{
    \medskip
    \begin{mdframed}[style=commandline]
}{
    \end{mdframed}
    \medskip
}

%----------------------------------------------------------------------------------------
%   FILE CONTENTS ENVIRONMENT
%----------------------------------------------------------------------------------------

% Usage:
% \begin{file}[optional filename, defaults to "File"]
%   File contents, for example, with a listings environment
% \end{file}

\mdfdefinestyle{file}{
    innertopmargin=1.6\baselineskip,
    innerbottommargin=0.8\baselineskip,
    topline=false, bottomline=false,
    leftline=false, rightline=false,
    leftmargin=2cm,
    rightmargin=2cm,
    singleextra={%
        \draw[fill=black!10!white](P)++(0,-1.2em)rectangle(P-|O);
        \node[anchor=north west]
        at(P-|O){\ttfamily\mdfilename};
        %
        \def\l{3em}
        \draw(O-|P)++(-\l,0)--++(\l,\l)--(P)--(P-|O)--(O)--cycle;
        \draw(O-|P)++(-\l,0)--++(0,\l)--++(\l,0);
    },
    nobreak,
}

% Define a custom environment for file contents
\newenvironment{file}[1][File]{ % Set the default filename to "File"
    \medskip
    \newcommand{\mdfilename}{#1}
    \begin{mdframed}[style=file]
}{
    \end{mdframed}
    \medskip
}

%----------------------------------------------------------------------------------------
%   NUMBERED QUESTIONS ENVIRONMENT
%----------------------------------------------------------------------------------------

% Usage:
% \begin{question}[optional title]
%   Question contents
% \end{question}

\mdfdefinestyle{question}{
    innertopmargin=1.2\baselineskip,
    innerbottommargin=0.8\baselineskip,
    roundcorner=5pt,
    nobreak,
    singleextra={%
        \draw(P-|O)node[xshift=1em,anchor=west,fill=white,draw,rounded corners=5pt]{%
        Question \theQuestion\questionTitle};
    },
}

\newcounter{Question} % Stores the current question number that gets iterated with each new question

% Define a custom environment for numbered questions
\newenvironment{question}[1][\unskip]{
    \bigskip
    \stepcounter{Question}
    \newcommand{\questionTitle}{~#1}
    \begin{mdframed}[style=question]
}{
    \end{mdframed}
    \medskip
}

%----------------------------------------------------------------------------------------
%   WARNING TEXT ENVIRONMENT
%----------------------------------------------------------------------------------------

% Usage:
% \begin{warn}[optional title, defaults to "Warning:"]
%   Contents
% \end{warn}

\mdfdefinestyle{warning}{
    topline=false, bottomline=false,
    leftline=false, rightline=false,
    nobreak,
    singleextra={%
        \draw(P-|O)++(-0.5em,0)node(tmp1){};
        \draw(P-|O)++(0.5em,0)node(tmp2){};
        \fill[black,rotate around={45:(P-|O)}](tmp1)rectangle(tmp2);
        \node at(P-|O){\color{white}\scriptsize\bf !};
        \draw[very thick](P-|O)++(0,-1em)--(O);%--(O-|P);
    }
}

% Define a custom environment for warning text
\newenvironment{warn}[1][Warning:]{ % Set the default warning to "Warning:"
    \medskip
    \begin{mdframed}[style=warning]
        \noindent{\textbf{#1}}
}{
    \end{mdframed}
}

%----------------------------------------------------------------------------------------
%   INFORMATION ENVIRONMENT
%----------------------------------------------------------------------------------------

% Usage:
% \begin{info}[optional title, defaults to "Info:"]
%   contents
%   \end{info}

\mdfdefinestyle{info}{%
    topline=false, bottomline=false,
    leftline=false, rightline=false,
    nobreak,
    singleextra={%
        \fill[black](P-|O)circle[radius=0.4em];
        \node at(P-|O){\color{white}\scriptsize\bf i};
        \draw[very thick](P-|O)++(0,-0.8em)--(O);%--(O-|P);
    }
}

% Define a custom environment for information
\newenvironment{info}[1][Info:]{ % Set the default title to "Info:"
    \medskip
    \begin{mdframed}[style=info]
        \noindent{\textbf{#1}}
}{
    \end{mdframed}
}


\newdimen\longformulasindent
\newenvironment{longformulas}
 {\global\longformulasindent=0pt
  \def\>{\global\advance\longformulasindent2em\relax\hspace{2em}}%
  \def\<{\global\advance\longformulasindent-2em\relax\hspace{-2em}}%
  \renewcommand{\arraystretch}{1.2}% some more room
  \begin{array}{@{}>{\displaystyle\hspace{\longformulasindent}}l@{}}}
 {\end{array}}

\newcommand{\code}{\texttt}
 % Include the file specifying the document structure and custom commands

\title{Algorithmik Blatt 1 Aufgabe 1} % Title of the assignment

\author{Mtr.-Nr. 6329857} % Author name and email address

\date{Universität Hamburg --- \today \\ Zusammengearbeitet mit 7330980} % University, school and/or department name(s) and a date

%----------------------------------------------------------------------------------------

\begin{document}

\maketitle % Print the title

\section*{Was macht der Algorithmus?}

\code{DoSomething(X, x, start, end)} sucht in dem Array \code{X} im Bereich zwischen Position
\code{start} und \code{end} nach dem Vorkommen des Elements, welches dem doppelten von \code{x}
entspricht, und gibt bei Erfolg die Position zurück, ansonsten \code{0}.

Dazu wird der zu durchsuchende Bereich des Arrays in drei Drittel geteilt und geprüft ob das gesuchte Element auf den inneren Grenzen der Bereiche --- sprich am Ende des ersten Drittels oder am Ende des zweiten Drittels --- liegt. Ist dies der Fall, ist das Ergebnis gefunden. Liegt das Gesuchte Element auf keiner dieser beiden Position, werden die drei Drittel jeweils nach dem gleichen Prinzip durchsucht, angefangen mit dem hinteren Drittel. Der Algorithmus bricht nicht ab, wenn ein Ergebnis gefunden wurde, sondern verarbeitet alle bereits geteilten Drittel --- sowie deren weitere Zerteilungen --- bis zu Ende.

\section*{Rekursive Form}

\begin{minted}[linenos]{text}

DoSomething(X, x, start, end)
    m_1 = floor(start +     (end - start) / 3) // 1/3 Position
    m_2 = floor(start + 2 * (end - start) / 3) // 2/3 Position
    if (start > end) // Rekursionsabbruch
        return 0
    if (X[m_1] == 2 * x) // Element gefunden
        return m_1
    else if (X[m_2] == 2 * x) // Element gefunden
        return m_2
    else
        return max( // Rekursion, nur einer kann >0 sein
            DoSomething(X, x, m_2 + 1, end    ), // (R1)
            DoSomething(X, x, m_1 + 1, m_2 - 1), // (R2)
            DoSomething(X, x, start  , m_1 - 1)  // (R3)
        )

\end{minted}

\section*{Rekurenzgleichung}

Der nichtrekursive Anteil der Funktion besteht nur aus einfachen arithmetischen Operationen, benötigt also konstant \code{c} viel Zeit. Die Eingabegröße $n$ ergibt sich aus der Differenz Parametern \code{start} und \code{end}, da diese den zu verarbeitenden Bereich angeben. Die Parameter \code{X} und \code{x} sind über die Rekursionstiefe konstant.

Der Algorithmus teilt das Problem in 3 Teilprobleme und reduziert die Problemgröße dabei von $n$ auf $\frac{n}{3} - 1$ bzw. auf $\frac{n}{3} - 2$. Denn sei $n = \text{end} - \text{start}$ ergibt sich:
\begin{itemize}
	\item für Aufruf $R_1$: $n = \text{end} - \left(m_2 + 1\right) = \left(\text{end} -
	\text{start}\right) / 3 - 1$

	\item für Aufruf $R_2$: $n = \left(m_2 - 1\right) - \left(m_1 + 1\right) = \left(\text{end} - \text{start}\right) / 3 - 2$
	\item und für Aufruf $R_3$: $n = \left(m_1 - 1 - \text{start}\right) = \left(\text{end} - \text{start}\right) / 3 - 1$.
\end{itemize}

Die Abbruchbedingung ist dass $\code{start} > \code{end}$, also dass $n < 0$ ist.

Daraus ergibt sich die Rekurenzgleichung:

\begin{equation*}
\begin{aligned}
T\left(n\right) & = c & \text{wenn $n < 0$}  \\
T\left(n\right) & = 3 \cdot T\left(n / 3 - q\right) + c & \text{wenn $n \ge 0$}
\end{aligned}
\end{equation*}

Wobei $q$ je nach Rekursionschritt 1 oder 2 ist, was aber, wie wir später sehen werden, keinen Unterschied macht.

Um das Mastertheorem verwenden zu können, bringen wir die Gleichung in die richtige Form:


\begin{equation*}
\begin{aligned}
\text{Sei $l$} & = n - 1&\\
T\left(l\right) & = c & \text{wenn $n = 0$}  \\
T\left(l\right) & = 3 \cdot T\left(l / 3 - q\right) + c & \text{wenn $l > 0$}
\end{aligned}
\end{equation*}

\begin{equation*}
\begin{aligned}
T\left(l\right) & = c & \text{wenn $l = 0$} \\
T\left(l\right) & = 3 \cdot T\left(\left(l - 3 \cdot q\right) / 3\right) + c & \text{wenn $l > 0$}
\end{aligned}
\end{equation*}



\begin{equation*}
\begin{aligned}
\text{Sei $m$} & = l - 3 \cdot q &\\
T\left(m + 3 \cdot q\right) &= 3 \cdot T\left(m / 3\right) + c  \\
S\left(m\right) & = T\left(m + 3 \cdot q\right) \\
S\left(m\right) & = 3 \cdot S\left(m / 3\right) + c
\end{aligned}
\end{equation*}

Nun hat die Rekurenzgleichung die Passende Form, um das Mastertheorem (MT)
anwenden zu können. Da die Parameter $a$ und $b$ des Mastertheorems 3 sind und der
nichtrekursive Teil konstant ist, tritt Fall 1 des MT ein, wonach der rekursive
Anteil dominiert und sich

$$
S\left(m\right) = \theta\left(N^{log_3{3}}\right)= \theta\left(m^{log_3{3}}\right) = \theta\left(m\right) = \theta\left(l - \left(3 \cdot q\right)\right) = \theta\left(n - 1 - \left(3 \cdot q\right)\right) = \theta\left(n\right)
$$

ergibt. Bisher hatten wir keinen Wert für $q$ gewählt, doch nun hat sich sowieso ergeben, dass er nur als konstanter Summand auftritt, also im $O$-Kalkül vernachlässigt werden kann.


\section*{Substitutionsmethode}

\begin{equation*}
    T\left(N, M\right) =
\begin{cases}
    1 & \text{für $N = 1$}\\
    T\left(N-1, M\right) + N & \text{für $1 < N \le M$} \\
    2 \cdot T\left(\frac{N}{2}, M\right) + N & \text{für $N > M$} \\
\end{cases}
\end{equation*}

\subsection*{Für $1 < N \le M$}


\begin{equation*}
\begin{aligned}
    T\left(N, M\right) & =  T\left(N-1, M\right) + N\\
& = T\left(N-2, M\right) + N + N\\
& = T\left(N-\left(N-1\right), M\right) + \left(N-1\right) \cdot N\\
& = T\left(1\right) + N^2 - N\\
& = 1 + N^2 - N\\
\end{aligned}
\end{equation*}

\subsection*{Für $N > M$}


\begin{equation*}
\begin{aligned}
    T(N, M) & =  2 \cdot T\left(\frac{N}{2}, M\right) + N\\
    & =  2 \cdot \left(2 \cdot T\left(\frac{N}{2}, M\right) + N\right) + N\\
    & =  4 \cdot T\left(\frac{N}{4}, M\right) + 2N + N\\
    & =  2^i \cdot T\left(\frac{N}{2^i}, M\right) + 2^{i-1}N + \cdots + N^{1}\\
    & =  2^i \cdot T\left(\frac{N}{2^i}, M\right) + \sum_{k=0}^{i-1} 2^kN \\
    i & \ge log_2\left(\frac{N}{M}\right) \text{ damit } \frac{N}{2^i} \le M \\
    i & = log_2\left(\frac{N}{M}\right) \text{ weil $N,M$ Zweierpotenz }
\end{aligned}
\end{equation*}
\begin{equation*}
\begin{aligned}
    T(N, M) & =  2^{log_2\left(\frac{N}{M}\right)} \cdot T\left(\frac{N}{2^{log_2\left(\frac{N}{M}\right)}}, \right) + \sum_{k=0}^{log_2\frac{N}{M}-1}{2^kN} \\
    & =  \frac{N}{M} \cdot T(\frac{N}{\frac{N}{M}}, M) + \sum_{k=0}^{log_2(\frac{N}{M})-1} 2^kN \\
    & =  \frac{N}{M} \cdot T(M, M) + \sum_{k=0}^{log_2\frac{N}{M}-1} 2^kN \\
    & =  \frac{N}{M} \cdot T(M, M) + \sum_{k=0}^{log_2(\frac{N}{M})-1} 2^kN \\
    & =  \frac{N}{M} \cdot T(M, M) + N \sum_{k=0}^{log_2\frac{N}{M}-1} 2^k \\
\end{aligned}
\end{equation*}
\begin{center}
Die Summe ist eine geometrische Reihe, also:
\end{center}
\begin{equation*}
\begin{aligned}
    T(N, M) & =  \frac{N}{M} \cdot T(M, M) + N \left(\frac{2^{log_2\frac{N}{M}} - 1}{1}\right) \\
    & =  \frac{N}{M} \cdot T(M, M) + N \left(\frac{N}{M} - 1\right) \\
    & =  \frac{N}{M} \cdot T(M, M) + \left(\frac{N^2}{M} - N\right)
\end{aligned}
\end{equation*}

\begin{center}
Ersetzen $T(M,M)$ durch $T(N,M)$ für $1 < N \le M$.
\end{center}

\begin{equation*}
\begin{aligned}
    T(N, M) & =  \frac{N}{M} \cdot \left(1 + M^2 - M\right) + \left(\frac{N^2}{M} - N\right) \\
    & =  \left(\frac{N}{M} + \frac{N}{M} \cdot M^2 - M \cdot \frac{N}{M}\right) + \left(\frac{N^2}{M} - N\right) \\
    & =  \left(\frac{N}{M} + NM - N\right) + \left(\frac{N^2}{M} - N\right) \\
    & =  \frac{N^2}{M} + \frac{N}{M} + NM - 2N \\
    & =  \frac{N^2 + N}{M} + N \left( M-2 \right) \\
\end{aligned}
\end{equation*}




\end{document}
