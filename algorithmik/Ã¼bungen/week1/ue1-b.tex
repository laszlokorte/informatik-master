\documentclass[parskip=half,a4paper]{scrartcl}

%%%%%%%%%%%%%%%%%%%%%%%%%%%%%%%%%%%%%%%%%
% Lachaise Assignment
% Structure Specification File
% Version 1.0 (26/6/2018)
%
% This template originates from:
% http://www.LaTeXTemplates.com
%
% Authors:
% Marion Lachaise & François Févotte
% Vel (vel@LaTeXTemplates.com)
%
% License:
% CC BY-NC-SA 3.0 (http://creativecommons.org/licenses/by-nc-sa/3.0/)
%
%%%%%%%%%%%%%%%%%%%%%%%%%%%%%%%%%%%%%%%%%

%----------------------------------------------------------------------------------------
%   PACKAGES AND OTHER DOCUMENT CONFIGURATIONS
%----------------------------------------------------------------------------------------

\usepackage[utf8]{inputenc} % Required for inputting international characters
\usepackage[T1]{fontenc} % Output font encoding for international characters

\usepackage{amsmath,stmaryrd} % Math packages

\usepackage{enumerate} % Custom item numbers for enumerations
\usepackage[shortlabels]{enumitem}
\usepackage{array}
\usepackage{lineno}
\usepackage{epstopdf}
\usepackage{graphics}

\usepackage[ruled]{algorithm2e} % Algorithms

\usepackage[framemethod=tikz]{mdframed} % Allows defining custom boxed/framed environments

\usepackage{minted}

\usepackage[utopia]{mathdesign}
\usepackage{csquotes}

\usepackage{listings} % File listings, with syntax highlighting
\lstset{
    basicstyle=\ttfamily, % Typeset listings in monospace font
    language=Java,
    numbers=left,
    stepnumber=1,
    showstringspaces=false,
    tabsize=3,
    breaklines=true,
    breakatwhitespace=false,
    frame=single,
    linewidth=\columnwidth
}

%----------------------------------------------------------------------------------------
%   DOCUMENT MARGINS
%----------------------------------------------------------------------------------------

\usepackage{geometry} % Required for adjusting page dimensions and margins

\geometry{
    paper=a4paper, % Paper size, change to letterpaper for US letter size
    top=2.5cm, % Top margin
    bottom=3cm, % Bottom margin
    left=2.5cm, % Left margin
    right=2.5cm, % Right margin
    headheight=14pt, % Header height
    footskip=1.5cm, % Space from the bottom margin to the baseline of the footer
    headsep=1.2cm, % Space from the top margin to the baseline of the header
    %showframe, % Uncomment to show how the type block is set on the page
}

%----------------------------------------------------------------------------------------
%   FONTS
%----------------------------------------------------------------------------------------

%----------------------------------------------------------------------------------------
%   COMMAND  ENVIRONMENT
%----------------------------------------------------------------------------------------

% Usage:
% \begin{commandline}
%   \begin{verbatim}
%       $ ls
%
%       Applications    Desktop ...
%   \end{verbatim}
% \end{commandline}

\mdfdefinestyle{commandline}{
    leftmargin=10pt,
    rightmargin=10pt,
    innerleftmargin=15pt,
    middlelinecolor=black!50!white,
    middlelinewidth=2pt,
    frametitlerule=false,
    backgroundcolor=black!5!white,
    frametitle={Command Line},
    frametitlefont={\normalfont\sffamily\color{white}\hspace{-1em}},
    frametitlebackgroundcolor=black!50!white,
    nobreak,
}

% Define a custom environment for command-line snapshots
\newenvironment{commandline}{
    \medskip
    \begin{mdframed}[style=commandline]
}{
    \end{mdframed}
    \medskip
}

%----------------------------------------------------------------------------------------
%   FILE CONTENTS ENVIRONMENT
%----------------------------------------------------------------------------------------

% Usage:
% \begin{file}[optional filename, defaults to "File"]
%   File contents, for example, with a listings environment
% \end{file}

\mdfdefinestyle{file}{
    innertopmargin=1.6\baselineskip,
    innerbottommargin=0.8\baselineskip,
    topline=false, bottomline=false,
    leftline=false, rightline=false,
    leftmargin=2cm,
    rightmargin=2cm,
    singleextra={%
        \draw[fill=black!10!white](P)++(0,-1.2em)rectangle(P-|O);
        \node[anchor=north west]
        at(P-|O){\ttfamily\mdfilename};
        %
        \def\l{3em}
        \draw(O-|P)++(-\l,0)--++(\l,\l)--(P)--(P-|O)--(O)--cycle;
        \draw(O-|P)++(-\l,0)--++(0,\l)--++(\l,0);
    },
    nobreak,
}

% Define a custom environment for file contents
\newenvironment{file}[1][File]{ % Set the default filename to "File"
    \medskip
    \newcommand{\mdfilename}{#1}
    \begin{mdframed}[style=file]
}{
    \end{mdframed}
    \medskip
}

%----------------------------------------------------------------------------------------
%   NUMBERED QUESTIONS ENVIRONMENT
%----------------------------------------------------------------------------------------

% Usage:
% \begin{question}[optional title]
%   Question contents
% \end{question}

\mdfdefinestyle{question}{
    innertopmargin=1.2\baselineskip,
    innerbottommargin=0.8\baselineskip,
    roundcorner=5pt,
    nobreak,
    singleextra={%
        \draw(P-|O)node[xshift=1em,anchor=west,fill=white,draw,rounded corners=5pt]{%
        Question \theQuestion\questionTitle};
    },
}

\newcounter{Question} % Stores the current question number that gets iterated with each new question

% Define a custom environment for numbered questions
\newenvironment{question}[1][\unskip]{
    \bigskip
    \stepcounter{Question}
    \newcommand{\questionTitle}{~#1}
    \begin{mdframed}[style=question]
}{
    \end{mdframed}
    \medskip
}

%----------------------------------------------------------------------------------------
%   WARNING TEXT ENVIRONMENT
%----------------------------------------------------------------------------------------

% Usage:
% \begin{warn}[optional title, defaults to "Warning:"]
%   Contents
% \end{warn}

\mdfdefinestyle{warning}{
    topline=false, bottomline=false,
    leftline=false, rightline=false,
    nobreak,
    singleextra={%
        \draw(P-|O)++(-0.5em,0)node(tmp1){};
        \draw(P-|O)++(0.5em,0)node(tmp2){};
        \fill[black,rotate around={45:(P-|O)}](tmp1)rectangle(tmp2);
        \node at(P-|O){\color{white}\scriptsize\bf !};
        \draw[very thick](P-|O)++(0,-1em)--(O);%--(O-|P);
    }
}

% Define a custom environment for warning text
\newenvironment{warn}[1][Warning:]{ % Set the default warning to "Warning:"
    \medskip
    \begin{mdframed}[style=warning]
        \noindent{\textbf{#1}}
}{
    \end{mdframed}
}

%----------------------------------------------------------------------------------------
%   INFORMATION ENVIRONMENT
%----------------------------------------------------------------------------------------

% Usage:
% \begin{info}[optional title, defaults to "Info:"]
%   contents
%   \end{info}

\mdfdefinestyle{info}{%
    topline=false, bottomline=false,
    leftline=false, rightline=false,
    nobreak,
    singleextra={%
        \fill[black](P-|O)circle[radius=0.4em];
        \node at(P-|O){\color{white}\scriptsize\bf i};
        \draw[very thick](P-|O)++(0,-0.8em)--(O);%--(O-|P);
    }
}

% Define a custom environment for information
\newenvironment{info}[1][Info:]{ % Set the default title to "Info:"
    \medskip
    \begin{mdframed}[style=info]
        \noindent{\textbf{#1}}
}{
    \end{mdframed}
}


\newdimen\longformulasindent
\newenvironment{longformulas}
 {\global\longformulasindent=0pt
  \def\>{\global\advance\longformulasindent2em\relax\hspace{2em}}%
  \def\<{\global\advance\longformulasindent-2em\relax\hspace{-2em}}%
  \renewcommand{\arraystretch}{1.2}% some more room
  \begin{array}{@{}>{\displaystyle\hspace{\longformulasindent}}l@{}}}
 {\end{array}}

\newcommand{\code}{\texttt}


\title{Algorithmik Blatt 1 Aufgabe 2}

\author{Mtr.-Nr. 6329857}

\date{Universität Hamburg --- \today \\ in Diskussion mit 7330980, 7327475, 7328242}

\begin{document}

\maketitle

\section*{Modulo Schleifeninvariante}

Als Schleifeninvariante gilt $\left(r = x - q \cdot y\right) \land \left(q \ge 0\right)$.

\subsection*{Initialisierung}

\begin{enumerate}
	\item $q = 0$, $r = x \implies \left(x = x - 0 \cdot y\right) \land \left(q \ge 0\right)$
\end{enumerate}

\subsection*{Fortsetzung}

\begin{enumerate}
	\item Angenommen zu Beginn einer Iteration gilt $r = x - q \cdot y$
	\item Die Schleife setzt $r' = r - y$ und $q' = q + 1$
	\item Umgeformt: $r = r' + y$ und $q = q' - 1$
	\item Die Schleife ändert weder $x$ noch $y$, also $x' = x$ und $y' = y$
	\item Einsetzen in Annahme: $r' + y' = x' - \left(q' - 1\right) \cdot y' = x' - q' \cdot y' + y'$
	\item Also gilt $r' = x' - q' \cdot y'$ und $q + 1 = q' \ge 0$ auch am Ende der Iteration.
\end{enumerate}

\subsection*{Terminierung}

Wenn die Schleife endet, ist $r < y$ und $x = q \cdot y + r$. Damit entspricht $r$ genau der Definition von $x \mod y$. Die Schleife terminiert wenn $y > 0$, weil $r$ in jeder Iteration um $y$ verringert wird.

\section*{Primzahl Schleifeninvarianz}

Als Schleifeninvarianten gelten:

\begin{enumerate}[a)]
	\item Bis $i^2$ sind nur Primzahlen markiert: $\forall p: \left(0 < p < i^2\right) \rightarrow \left(P[p] \rightarrow \forall z: \left(z \le n\right) \rightarrow
	\neg P[p \cdot z]\right)$

	\item Alle Primzahlen im bis $i^2$ sind markiert: $\forall p: \left(0 < p < i^2\right) \rightarrow \left(\text{isPrime}\left(p\right) \rightarrow P[p]\right)$
\end{enumerate}

Sie ließen sich auch in einer gemeinsamen Formel ausdrücken, allerdings ist es übersichtlicher sie getrennt zu zeigen

\subsection*{Initialisierung (a)}

Zu beginn ist $P[1] = \code{false}$ und $i = 2$. Als einziges $p$ ist 1 zu prüfen. Das Antezedens des inneren Konditionals ($P[1]$) ist falsch, also gilt die Invariante.

\subsection*{Initialisierung gb)}

Da das Array mit \code{true} initialisiert wird, gilt das Konditional in Invariante b zu Beginn für alle $p$.

\subsection*{Fortsetzung (a)}

Die Schleife erhöht $i$ in jedem Durchlauf um $1$. Die Invariante muss also nach jedem Durchlauf für $i$ weitere $p$ gelten sowie auch für alle niedrigeren $p$, für die sie schon im Vorherigen Durchlauf galt. Bis auf das Inkrement wird $i$ nicht verändert.

Die Invariante besteht im Kern aus einem Konditional. Sie könnte also nur dadurch verletzt werden, dass das Antezedens wahr, aber das Konsequenz falsch ist. Um die Invariante zu verletzen, müsste also gelten, dass es ein $p < i^2$ und ein $z \le n$ gibt für die gilt $P[p] \land P[p \cdot z]$.

In jedem Durchlauf der äußeren Schleife wird aber nur $P[k], k \ge i^2$ beschrieben. Der linke Operand kann also durch die Operationen der Schleife nicht wahr werden. Außerdem wird das Array nur mit \code{false} Werten beschrieben, also kann auch der rechte Operand nicht wahr werden. Es ist also nicht möglich eine Belegung zu finden, die die Invariante verletzt.

\subsection*{Fortsetzung (b)}

Der Algorithmus beschreibt nur Felder des Arrays, deren Index ein vielfaches ($>1$) von $i$ sind. Also wird das Konsequenz des Konditionals von Invariante b niemals mit $false$ belegt.

\subsection*{Terminierung (a)}

Die Schleife terminiert mit $i^2 > n$, also gilt $\forall p:\left( 0 < p < n\right) \rightarrow \left(P[p] -> \forall z: \left(z \le n\right) \rightarrow \left(\neg P[p \cdot z]\right)\right)$. Der Algorithmus hat also alle Felder mit \code{false} belegt, deren Index ein vielfaches eines Index $B$ ist, dessen zugehöriges Feld $P[B] = \code{true}$ ist.

\subsection*{Terminierung (b)}

Da im Verlauf nur Felder mit ganzzahlig-vielfachen Indizes beschrieben wurden, sind alle Primzahlfelder immer noch mit der initialen Belegung von \code{true}
belegt.

\subsection*{Terminierung}

Also sind genau die Primzahlfelder im Array mit \code{true} und alle anderen mit
\code{false} markiert.

\end{document}
