\documentclass[parskip=half,a4paper]{scrartcl}
\usepackage[ngerman]{babel}


%%%%%%%%%%%%%%%%%%%%%%%%%%%%%%%%%%%%%%%%%
% Lachaise Assignment
% Structure Specification File
% Version 1.0 (26/6/2018)
%
% This template originates from:
% http://www.LaTeXTemplates.com
%
% Authors:
% Marion Lachaise & François Févotte
% Vel (vel@LaTeXTemplates.com)
%
% License:
% CC BY-NC-SA 3.0 (http://creativecommons.org/licenses/by-nc-sa/3.0/)
%
%%%%%%%%%%%%%%%%%%%%%%%%%%%%%%%%%%%%%%%%%

%----------------------------------------------------------------------------------------
%   PACKAGES AND OTHER DOCUMENT CONFIGURATIONS
%----------------------------------------------------------------------------------------

\usepackage[utf8]{inputenc} % Required for inputting international characters
\usepackage[T1]{fontenc} % Output font encoding for international characters

\usepackage{amsmath,stmaryrd} % Math packages

\usepackage{enumerate} % Custom item numbers for enumerations
\usepackage[shortlabels]{enumitem}
\usepackage{array}
\usepackage{lineno}
\usepackage{epstopdf}
\usepackage{graphics}

\usepackage[ruled]{algorithm2e} % Algorithms

\usepackage[framemethod=tikz]{mdframed} % Allows defining custom boxed/framed environments

\usepackage{minted}

\usepackage[utopia]{mathdesign}
\usepackage{csquotes}

\usepackage{listings} % File listings, with syntax highlighting
\lstset{
    basicstyle=\ttfamily, % Typeset listings in monospace font
    language=Java,
    numbers=left,
    stepnumber=1,
    showstringspaces=false,
    tabsize=3,
    breaklines=true,
    breakatwhitespace=false,
    frame=single,
    linewidth=\columnwidth
}

%----------------------------------------------------------------------------------------
%   DOCUMENT MARGINS
%----------------------------------------------------------------------------------------

\usepackage{geometry} % Required for adjusting page dimensions and margins

\geometry{
    paper=a4paper, % Paper size, change to letterpaper for US letter size
    top=2.5cm, % Top margin
    bottom=3cm, % Bottom margin
    left=2.5cm, % Left margin
    right=2.5cm, % Right margin
    headheight=14pt, % Header height
    footskip=1.5cm, % Space from the bottom margin to the baseline of the footer
    headsep=1.2cm, % Space from the top margin to the baseline of the header
    %showframe, % Uncomment to show how the type block is set on the page
}

%----------------------------------------------------------------------------------------
%   FONTS
%----------------------------------------------------------------------------------------

%----------------------------------------------------------------------------------------
%   COMMAND  ENVIRONMENT
%----------------------------------------------------------------------------------------

% Usage:
% \begin{commandline}
%   \begin{verbatim}
%       $ ls
%
%       Applications    Desktop ...
%   \end{verbatim}
% \end{commandline}

\mdfdefinestyle{commandline}{
    leftmargin=10pt,
    rightmargin=10pt,
    innerleftmargin=15pt,
    middlelinecolor=black!50!white,
    middlelinewidth=2pt,
    frametitlerule=false,
    backgroundcolor=black!5!white,
    frametitle={Command Line},
    frametitlefont={\normalfont\sffamily\color{white}\hspace{-1em}},
    frametitlebackgroundcolor=black!50!white,
    nobreak,
}

% Define a custom environment for command-line snapshots
\newenvironment{commandline}{
    \medskip
    \begin{mdframed}[style=commandline]
}{
    \end{mdframed}
    \medskip
}

%----------------------------------------------------------------------------------------
%   FILE CONTENTS ENVIRONMENT
%----------------------------------------------------------------------------------------

% Usage:
% \begin{file}[optional filename, defaults to "File"]
%   File contents, for example, with a listings environment
% \end{file}

\mdfdefinestyle{file}{
    innertopmargin=1.6\baselineskip,
    innerbottommargin=0.8\baselineskip,
    topline=false, bottomline=false,
    leftline=false, rightline=false,
    leftmargin=2cm,
    rightmargin=2cm,
    singleextra={%
        \draw[fill=black!10!white](P)++(0,-1.2em)rectangle(P-|O);
        \node[anchor=north west]
        at(P-|O){\ttfamily\mdfilename};
        %
        \def\l{3em}
        \draw(O-|P)++(-\l,0)--++(\l,\l)--(P)--(P-|O)--(O)--cycle;
        \draw(O-|P)++(-\l,0)--++(0,\l)--++(\l,0);
    },
    nobreak,
}

% Define a custom environment for file contents
\newenvironment{file}[1][File]{ % Set the default filename to "File"
    \medskip
    \newcommand{\mdfilename}{#1}
    \begin{mdframed}[style=file]
}{
    \end{mdframed}
    \medskip
}

%----------------------------------------------------------------------------------------
%   NUMBERED QUESTIONS ENVIRONMENT
%----------------------------------------------------------------------------------------

% Usage:
% \begin{question}[optional title]
%   Question contents
% \end{question}

\mdfdefinestyle{question}{
    innertopmargin=1.2\baselineskip,
    innerbottommargin=0.8\baselineskip,
    roundcorner=5pt,
    nobreak,
    singleextra={%
        \draw(P-|O)node[xshift=1em,anchor=west,fill=white,draw,rounded corners=5pt]{%
        Question \theQuestion\questionTitle};
    },
}

\newcounter{Question} % Stores the current question number that gets iterated with each new question

% Define a custom environment for numbered questions
\newenvironment{question}[1][\unskip]{
    \bigskip
    \stepcounter{Question}
    \newcommand{\questionTitle}{~#1}
    \begin{mdframed}[style=question]
}{
    \end{mdframed}
    \medskip
}

%----------------------------------------------------------------------------------------
%   WARNING TEXT ENVIRONMENT
%----------------------------------------------------------------------------------------

% Usage:
% \begin{warn}[optional title, defaults to "Warning:"]
%   Contents
% \end{warn}

\mdfdefinestyle{warning}{
    topline=false, bottomline=false,
    leftline=false, rightline=false,
    nobreak,
    singleextra={%
        \draw(P-|O)++(-0.5em,0)node(tmp1){};
        \draw(P-|O)++(0.5em,0)node(tmp2){};
        \fill[black,rotate around={45:(P-|O)}](tmp1)rectangle(tmp2);
        \node at(P-|O){\color{white}\scriptsize\bf !};
        \draw[very thick](P-|O)++(0,-1em)--(O);%--(O-|P);
    }
}

% Define a custom environment for warning text
\newenvironment{warn}[1][Warning:]{ % Set the default warning to "Warning:"
    \medskip
    \begin{mdframed}[style=warning]
        \noindent{\textbf{#1}}
}{
    \end{mdframed}
}

%----------------------------------------------------------------------------------------
%   INFORMATION ENVIRONMENT
%----------------------------------------------------------------------------------------

% Usage:
% \begin{info}[optional title, defaults to "Info:"]
%   contents
%   \end{info}

\mdfdefinestyle{info}{%
    topline=false, bottomline=false,
    leftline=false, rightline=false,
    nobreak,
    singleextra={%
        \fill[black](P-|O)circle[radius=0.4em];
        \node at(P-|O){\color{white}\scriptsize\bf i};
        \draw[very thick](P-|O)++(0,-0.8em)--(O);%--(O-|P);
    }
}

% Define a custom environment for information
\newenvironment{info}[1][Info:]{ % Set the default title to "Info:"
    \medskip
    \begin{mdframed}[style=info]
        \noindent{\textbf{#1}}
}{
    \end{mdframed}
}


\newdimen\longformulasindent
\newenvironment{longformulas}
 {\global\longformulasindent=0pt
  \def\>{\global\advance\longformulasindent2em\relax\hspace{2em}}%
  \def\<{\global\advance\longformulasindent-2em\relax\hspace{-2em}}%
  \renewcommand{\arraystretch}{1.2}% some more room
  \begin{array}{@{}>{\displaystyle\hspace{\longformulasindent}}l@{}}}
 {\end{array}}

\newcommand{\code}{\texttt}

\usetikzlibrary{positioning}
\usetikzlibrary{arrows.meta}
\title{Algorithmik Blatt 5 Teil 2}

\author{Mtr.-Nr. 6329857}

\date{Universität Hamburg --- \today}

\begin{document}

\maketitle % Print the title
\linenumbers

\section*{Aufgabe 14}

\subsection*{Teil 1}


\begin{tikzpicture}
\begin{scope}[every node/.style={circle,thick,draw}]
\node (S) {$S$};
\node (A) [right = 2cm  of S]{$A$};
\node (B) [right = 2cm  of A]{$B$};
\node (C) [right = 2cm  of B]{$C$};
\end{scope}

\begin{scope}[>={Stealth[black]},
              every node/.style={fill=white,circle},
              every edge/.style={draw=black,very thick}]
\draw[->] (S) edge node[above] {1} (A);
\draw[->] (A) edge node[above] {1} (B);
\draw[->] (B) edge node[above] {1} (C);
\draw[->] (S) edge[bend left=30] node[above] {10} (C);
\end{scope}

\end{tikzpicture}

Wir hätten gerne alle kürzesten Pfade der Länge $k=1$.

Reihenfolge der Relaxion: $(S,A), (A,B), (B,C), (S,C)$. Nach einer Iteration von Bellman-Ford ist von $S$ nach $C$ bereits der wirklich kürzeste Pfad Gefunden. Allerdings hat dieser Pfad $3$ Kanten lang. Wir hatten uns aber nur für Pfade der Länge $1$ interessiert. Was Bellman-Ford korrekt bestimmt hat, ist die Länge der $k$ langen Prefixe der insgesamt kürzesten Pfade.

\subsection*{Teil 2}

Das Problem ist, dass bei zu guter Wahl der Reihenfolge der Relaxion nach der $n$-ten Iteration bereits Pfade gefunden wurden, die länger sind als $n$. Wir wolle aber nach der $n$-ten Iteration nur Entfernungen bestimmt haben, die weniger als $n$ Kanten entfernt liegen. Dann können wir sicher sein, dass die Knoten, deren Entfernung wir kennen, nur über Pfade der Länge $n$ erreicht wurden --- somit die bekannten Entfernungen sich auch nur auf diese Pfade beziehen.

In jeder Iteration darf die Kantenzahl der längsten bekannten Pfade nur um 1 erhöhnt werden. Ausgehende Kanten von Knoten, die selbst in der aktuellen Iteration erst erreicht wurden (Distanz wurde von $\infty$ auf eine Zahl gesetzt), müssen bis zum Ende der Iteration selbst von der Relaxion ausgeschlossen werden.


\section*{Aufgabe 15}


\begin{algorithmic}[1]
\Procedure{ClosestTreeLeafs}{V, E, W} \Comment{Knoten, Kanten, Gewichte}
\State $(E',V') = \text{Kruskal}(E, V)$ \Comment{$\mathcal{O}(|E|\log(|V|))$}
\State $\text{queue} = \text{initQueue}()$
\State bestLeafs = InitArray(|V|)
\State bestDistance = InitArray(|V|)
\For{n}{0}{$|V|$} \Comment{$\mathcal{O}(|V|)$}
\If{\text{deg}(V[i]) == 1} \Comment{Leaf Node}
\State bestLeafs[V[i]] = V[i]
\State bestDistance[V[i]] = 0
\State insertQueue(V[i], queue, bestDistance[V[i]]) \Comment{$\mathcal{O}(1)$, weil bestDistance[i] $= 0$}
\Else \Comment{Inner Node}
\State bestLeafs[V[i]] = nil
\State bestDistance[V[i]] = $\infty$
\EndIf
\EndFor
\While{currentNode = extractMin(queue)}  \label{outer} \Comment{Zeile \ref{outer} und \ref{inner} gesamt $\mathcal{O}(|E|)$}
\While{targetNode = selectOutgoingEdge(E', currentNode)} \label{inner}
\State removeEdge(E', currentNode, targetNode)
\State newDistance = bestDistance[currentNode] + W((currentNode, targetNode))
\If{bestDistance[targetNode] == $\infty$}
\State bestDistance[targetNode] = newDistance
\State insertQueue(queue, targetNode, newDistance) \Comment{ $\mathcal{O}(\log(|V|))$}
\ElsIf{newDistance < bestDistance[targetNode]}
\State bestDistance[targetNode] = newDistance
\State decreaseKey(queue, targetNode, newDistance) \Comment{ $\mathcal{O}(\log(|V|))$}
\EndIf
\EndWhile
\EndWhile
\Return bestLeafs
\EndProcedure
\end{algorithmic}


\subsection*{Argumentation}

Wir nehmen an, dass sich aus der Fernreisekarte in linearer Zeit ein Graph konstruieren lässt, in welchem jede Stadt und jeder Hafen einen von einem Knoten dargestellt wird und genau jede Kante einer Stra{\ss}e entspricht die die jeweiligen Städte mit einander bzw mit den erreichbaren Häfen verbindet.

Wir gehen davon aus, dass dieser Graph $G$ zusammenhängend ist und keine negativen Kanten enthält. Daher können wir aus ihm einen minimalen Spannbaum konstruieren. Mit dem Algorithmus von Kruskal lässt sich der minimale Spannbaum in \(\mathcal{O}(|E|\log(|V|))\) konstruieren.

Da die Häfen nicht miteinander verbunden sind, jede Stadt aber mit mindestens einem Hafen verbunden ist (Zusammenhang), sind genau die Häfen Blattknoten im Spannbaum.

Als nächtes durchlaufen wir den Baum Ebene für Ebene beginnend bei den Blättern (Häfen) und weisen dabei jedem Knoten den, nächsten Blattknoten, sprich den besten Vorgänger ($\text{Pre}$), und die Distanz zu diesem zu. Das passiert wie folgt:

\begin{enumerate}
    \item Alle Blattknoten bekommen sich selbst als besten Vorgänger zugewiesen $\text{Pre}(k) = k$
    \item Alle Blattknoten $k$ bekommen die Distanz $D(k) = 0$
    \item Alle inneren Knoten $k$ bekommen initial die Distanz $D(k) = \infty$
    \item Alle inneren Knoten $k$ bekommen initial als besten Vorgänger $\text{Pre} = \text{nil}$
    \item Wird ein Knoten $q$ über eine Kante $(p,q)$ Besucht, prüfe ob $W(p,q) + D(p) < D(q)$, wenn ja aktualisiere $\text{Pre}(q) = p$ und $D(q) = W(p,q) + D(p)$
    \item Wenn für einen Knoten der beste Vorgänger und die niedrigste Distanz zu diesem bestimmt ist, kann das Verfahren von diesem Knoten aus wiederholt werden.
\end{enumerate}

Die Intuition hinter diesem Verfahren ist, an jedem Hafen gleichzeitig einen Radius von 0 an mit gleicher Geschwindigkeit wachsen zu lassen und jede Stadt dem Hafen zuzuordnen, dessen Radius sie zu erst erreicht. In einem diskreten Verfahren auf unserem Graphen muss dieses gleichzeitig Wachstum aber explizit sichergestellt werden. Wir wollen zudem sicherstellen, dass jede Kante nur einmal besucht werden muss.

Um das gleichzeitige Wachstum zu erreichen, ordnen wir die bereits erreichten Knoten (zu Beginn alle Häfen) in einer Warteschlange nach ihrer bisher besten Distanz. Zu einem Knoten, der den Kopf der Schlange erreicht, kann es keinen kürzeren Pfad von einem Hafen aus geben, weil alle anderen Knoten in der Schlange bereits eine grö{\ss}ere Distanz zum nächstens Hafen haben. Alle Knoten die noch nicht in der Schlage sind, haben mindestens eine beste Distanz so gro{\ss} wie die Knoten, die schon in der Schlange sind. Daraus ergibt sich, dass wenn wir eine Kante $(p,q)$ Benutzen, wir diese nicht noch ein weiteres mal betrachten müssen, weil sich die beste Distanz von $p$ nicht mehr ändert und diese beste Distanz von $q$ aus Sicht dieser Kante auch nicht mehr verbessern kann. Die beste Distanz von $q$ kann nur dadurch besser werden, dass wir über eine andere Kante $(r,q)$ nach $q$ gelangen. Dafür muss $r$ eine kürzere beste Distanz haben als $q$, sodass wir auch sicher sein können eine solche Kante benutzt zu haben bevor $q$ an den Kopf der Warteschlange rutscht.

Bei jedem Benutzen einer Kante müssen wir potentiell die beste Distanz des Zielknotens aktualisieren und auch die Position des Knotens in der Warteschlange aktualisieren. Der Aufwand dafür ist in \(\mathcal{O}(\log(l))\), wobei $l$ die Länge der Warteschlange, also maximal die Anzahl der Knoten ist. Daraus ergibt sich als Kosten für das Durchlaufen des Baums \(\mathcal{O}(|E|\log(|V|))\)

Da wir uns in einem Baum bewegen, und von jedem Knoten aus alle nicht benutzten Kanten benutzen, können wir sicher sein, am Ende alle Kanten benutzt und alle Knoten besucht zu haben. Also hat auch jeder Knoten seinen besten Hafen und die kürzeste Distanz zu diesem zugewiesen bekommen.



\end{document}
